\addcontentsline{toc}{chapter}{ABSTRAK}
\thispagestyle{plain}
\begin{centering}
	\textbf{\MakeUppercase{\judul}}
\end{centering} \\[20pt]
\begin{tabular}{ll}
	Nama Mahasiswa  &: \MakeUppercase{\namaPenulisSatu} \\
	NRP &: \nrpPenulisSatu \\
	Nama Mahasiswa  & : \MakeUppercase{\namaPenulisDua} \\
	NRP &: \nrpPenulisDua \\
	Departemen  &: Departemen \departemen FTIK-ITS \\
	Pembimbing Departemen  &: \pembimbingDept \\
	Pembimbing Lapangan  &: \pembimbingLap
\end{tabular} \\[20pt]
\begin{centering}
	\textbf{Abstrak}
\end{centering}
{\itshape
\\
\indent Dengan kemajuan teknologi di zaman sekarang, BRI menjadi satu-satunya Bank di dunia yang memiliki satelit sendiri. Oleh karenanya, dilakukan relokasi ATM untuk diarahkan langsung menuju satelit milik BRI. Untuk keberlangsungan kegiatan relokasi ATM, diperlukan sebuah sistem informasi sebagai wadah kontrol pengerjaan proyek relokasi ATM BRI.\\
Pada laporan Kerja Praktik kali ini, penulis akan menguraikan secara garis besar pengerjaan aplikasi \textit{Monitoring} SIK yang menggunakan bahasa pemrograman HTML, CSS, JavaScript serta PHP.\\
Berdasarkan hasil uji coba dan evaluasi menunjukkan bahwa aplikasi \textit{Monitoring} SIK yang dibuat telah berhasil memenuhi kebutuhan informasi yang dibutuhkan dalam memantau pengerjaan proyek relokasi.
\rm \\
\textbf{Kata Kunci: \textit{monitoring}, SIK, relokasi}


\cleardoublepage
