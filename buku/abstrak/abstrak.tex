\addcontentsline{toc}{chapter}{ABSTRAK}
\thispagestyle{plain}
\begin{centering}
	\textbf{\MakeUppercase{\judul}}
\end{centering} \\[20pt]
\begin{tabular}{ll}
	Nama Mahasiswa  &: \MakeUppercase{\namaPenulisSatu} \\
	NRP &: \nrpPenulisSatu \\
	Nama Mahasiswa  & : \MakeUppercase{\namaPenulisDua} \\
	NRP &: \nrpPenulisDua \\
	Departemen  &: Departemen \departemen FTIK-ITS \\
	Pembimbing Departemen  &: \pembimbingDept \\
	Pembimbing Lapangan  &: \pembimbingLap
\end{tabular} \\[10pt]
\begin{center}
	\textbf{ABSTRAK}
\end{center}
\indent\indent PT Telkom Indonesia (Persero) Tbk (Telkom) adalah Badan Usaha Milik Negara (BUMN) yang bergerak di bidang jasa layanan teknologi informasi dan komunikasi (TIK) dan jaringan telekomunikasi di Indonesia. Saat ini, Telkom sedang mengembangkan sebuah produk yang dapat mengimplementasikan \textit{server-side Internet of Things} (IoT) \textit{platform} bernama WSO2 IoT Server. Sebelum produk tersebut diperjualbelikan, Telkom memerlukan sebuah sistem yang dapat menguji coba WSO2 IoT Server, sekaligus memiliki daya guna lebih, yakni untuk meningkatkan efisiensi persiapan ruang rapat di PT Telkom Indonesia. Oleh karena itu, penulis membuat sebuah \namaSistem. \\
\indent Pada laporan kerja praktik ini, penulis akan menguraikan secara garis besar pengerjaan \namaSistem menggunakan mikrokontroler Arduino UNO, beberapa komponen elektronika seperti sensor suhu dan infrared, \textit{breadboard}, dan \textit{jumper wires}, Arduino IDE dan bahasa pemrograman C, serta memanfaatkan Telegram Bot API.\\
\indent Berdasarkan hasil uji coba dan evaluasi menunjukkan bahwa \namaSistem yang dibuat telah berhasil memenuhi kebutuhan efisiensi persiapan ruang rapat di PT Telekomunikasi Indonesia. \\[10pt]
\textbf{Kata Kunci: \textit{Internet of Things}, \textit{EasyMeeting}, Mikrokontroler, Arduino UNO, Telegram Bot, Rapat}


\cleardoublepage
