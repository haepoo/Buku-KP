\begin{thebibliography}{9}
	
	\bibitem{arduino-uno}
	Arduino. (2018). \textit{Arduino UNO \& Genuino UNO}. [online] Diakses pada 23 Juni 2018, dari http://arduino.cc/en/Main/ArduinoBoardUno/
		
	\bibitem{iot-rfidjournal}
	Ashton, K. (2009, 22 Juni). That 'Internet of Things' Thing. \textit{RFID Journal}. [online] Diakses pada 22 Juni 2018, dari http://www.rfidjournal.com/articles/view?4986
	
	\bibitem{iot-technopedia}
	Janssen, D. \& Janssen, C. (n.d.). \textit{Internet of Things (IoT)}. [online] Diakses pada 22 Juni 2018, dari https://www.techopedia.com/definition/28247/internet-of-things-iot
	
	\bibitem{Perkembangan Industri Internet of Things di Indonesia Tahun 2017}
	Ryza, P. (2017, 20 Desember). 
	\textit{Perkembangan Industri Internet of Things di Indonesia Tahun 2017}. [online] Diakses pada 02 Juli 2018, dari https://dailysocial.id/post/perkembangan- industri-internet-of-things-di-indonesia-tahun-2017
	
	\bibitem{telegram}
	Telegram. (n.d.). \textit{Telegram APIs}. [online] Diakses pada 23 juni 2018, dari https://core.telegram.org/api
	
	\bibitem{telegram-bot}
	Telegram Team. (2015). \textit{Telegram Bot Platform}. [online] Diakses pada 23 juni 2018, dari https://telegram.org/blog/bot-revolution
	
	\bibitem{soa}
	Telkom Indonesia. (2017, 11 Oktober). \textit{SOA : Untying the Knot, Information Technology Division} [powerpoint slides]. Teks tidak terpublikasi, Telkom Indonesia, Jakarta Selatan, DKI Jakarta, Indonesia. 
	
	\bibitem{Profil Telkom}	
	Telkom Indonesia. (n.d.). \textit{Profil dan Riwayat Singkat}. [online] Diakses pada 19 Juni 2018, dari  https://www.telkom.co.id/ servlet/tk/about/id\_ID/stocklanding/profil-dan-riwayat-singkat.html
		
\end{thebibliography}
