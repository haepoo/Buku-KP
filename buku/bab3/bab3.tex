\chapter{TINJAUAN PUSTAKA}
\section{\textit{Internet of Things}}
\tab \textit{Internet of Things} (IoT) adalah sebuah konsep komputasi yang menggambarkan sebuah gagasan tentang objek fisik sehari-hari yang terhubung ke internet dan mampu mengidentifikasi diri mereka ke perangkat lain. Istilah ini erat diidentifikasi oleh RFID sebagai metode komunikasi antar objek tanpa melibatkan interaksi manusia ke manusia atau manusia ke komputer, termasuk teknologi sensor, \textit{wireless}, atau \textit{QR codes}. IoT telah berkembang dari konvergensi teknologi nirkabel, \textit{micro-electromechanical systems} (MEMS), dan internet.\\
\tab IoT adalah signifikan, karena sebuah objek dapat merepresentasikan dirinya secara digital menjadi sesuatu yang lebih besar dari objek itu sendiri. Objek tidak hanya bisa berhubungan dengan pengguna, namun objek dapat terhubung dengan objek lain disekitarnya dan dengan basis data. Ketika banyak objek bekerja bersama-sama sebagai satu kesatuan, maka mereka memiliki kecerdasan yang disebut \textit{"ambient intelligence"}.\\
\tab \textit{Internet of Things} adalah konsep yang sulit didefinisikan secara tepat sebab penelitian pada IoT masih dalam tahap perkembangan. Bahkan ada banyak kelompok yang berbeda yang telah mendefinisikan istilah tersebut, meski penggunaan awal istilah ini dikaitkan dengan Kevin Ashton, seorang pakar inovasi digital. Ada sebuah pernyataan Ashton (1999) yang dikutip dari sebuah artikel di \textit{RFID Journal}:\\
\tab \textit{"Jika kita memiliki komputer yang tahu segalanya yang harus diketahui dari berbagai benda \textit{(things)} – menggunakan data yang mereka kumpulkan sendiri tanpa bantuan dari kita – kita bisa melacak dan menghitung semuanya, serta sangat mengurangi pemborosan, kerugian, dan biaya. Kita bisa tahu kapan suatu benda perlu diganti dan diperbaiki, atau kapan benda-benda itu masih dalam kondisi baik."}\\
\tab IoT menggambarkan sebuah dunia dimana hampir semua hal dapat terhubung dan berkomunikasi. Dengan kata lain, dengan \textit{Internet of Things}, dunia akan menjadi satu kesatuan sistem informasi yang sangat besar. \textit{Internet of Things} memiliki potensi untuk mengubah dunia seperti pernah dilakukan oleh internet, bahkan mungkin lebih baik (Ashton, 2009). 

\section{Mikrokontroler}
\subsection{Modul Ultrasonic}

\section{NodeMCU}
\tab NodeMCU adalah sebuah \textit{platform} IoT yang bersifat \textit{opensource}, terdiri dari perangkat keras berupa \textit{System On Chip} ESP8266 dan \textit{firmware} yang menggunakan bahasa pemrograman \textit{scripting} Lua (eLua). ESP8266 adalah nama mikrokontroler yang dirancang oleh \textit{Espressif Systems}, merupakan sebuah solusi jaringan WiFi yang menjembatani antara mikrokontroler yang sudah ada supaya bisa terhubung dengan WiFi dan juga dapat menjalankan aplikasi secara mandiri. Istilah "NodeMCU" secara \textit{default} sebenarnya mengacu pada \textit{firmware} yang digunakan, bukan pada perangkat keras \textit{development kit}.\\
\tab NodeMCU bisa dianalogikan sebagai \textit{board} arduino-nya ESP8266. NodeMCU telah mem-\textit{package} ESP8266 ke dalam sebuah \textit{board} yang kompak dengan berbagai fitur layaknya mikrokontroler dan kapabilitas akses terhadap Wifi, juga \textit{chip} komunikasi \textit{USB to serial}. Sehingga untuk memprogramnya hanya diperlukan ekstensi kabel data USB persis yang digunakan sebagai kabel data dan kabel \textit{charging smartphone} Android.

\subsection{Spesifikasi}
\begin{tabular}{ll}
	\tab \textit{Voltag}e &: 3.3V\\
	\tab GPIOs &: 17 \textit{(multiplexed with other functions)}\\
	\tab\textit{WiFi Mode} &: \textit{Direct} (P2P), \textit{soft-AP}\\
	\tab \textit{WiFi Protocols} &: 802.11 \textit{support} b/g/n\\
	\tab \textit{Current consumption} &: 10uA~170mA\\
	\tab \textit{Flash memory attachable} &: 16MB max (512K normal)\\
	\tab \textit{Network Protocols} &: \textit{Integrated} TCP/IP \textit{protocol stack}\\
	\tab M\textit{aximum concurrent}  &: 5\\
	\tab \textit{TCP connections}\\
	\tab \textit{Processor} &: Tensilica L106 32-bit\\
	\tab \textit{Processor speed }&: 80~160MHz\\
	\tab RAM &: 32K + 80K\\
	\tab \textit{Analog to Digital} &: 1 \textit{input with} 1024 \textit{step resolution}\\
\end{tabular}
	
\subsection{Skematik Posisi Pin}


\section{Arduino IDE}





\section{Telegram}
\tab Telegram adalah sebuah aplikasi \textit{messaging} yang fokus pada kecepatan dan keamanan, sederhana, ringan, dan juga gratis. Telegram dapat digunakan di semua perangkat pada waktu yang bersamaan dan pesan akan tersinkron dengan baik. \textit{Instant Messaging} (IM) Telegram yang diluncurkan pada Agustus tahun 2013, menjadi salah satu aplikasi IM yang banyak digunakan oleh masyarakat di seluruh dunia. Kelebihan
IM Telegram salah satunya adalah adanya landasan untuk
menggunakan \textit{Application Programming Interface} (API) untuk
masyarakat luas.

\subsection{Telegram APIs}
\tab Telegram menyediakan 2 jenis API yang dapat dikembangkan. API yang pertama adalah Telegram API, yakni API yang memungkinkan siapa saja membuat klien Telegram mereka sendiri sesuai dengan keinginan. Telegram adalah aplikasi yang \textit{open source} karena \textit{source code}-nya disebarluaskan secara luas dan gratis untuk dipelajari dan dibuka 100\% untuk para pengembang yang ingin membuat aplikasi Telegram menggunakan platform mereka. \\
\tab API yang kedua adalah Bot API, yakni API yang memungkinkan siapa saja dengan mudah membuat sebuah program yang menggunakan aplikasi perpesanan Telegram sebagai antarmuka. Bot API memungkinkan siapa saja untuk membuat bot (kependekan dari robot) yang dapat membalas dan melayani semua pengguna yang mengirimkan suatu pesan perintah kepada bot tersebut.
\subsection{Telegram Bot API}
\tab Bot adalah aplikasi pihak ketiga yang dijalankan di dalam Telegram dan dioperasikan oleh perangkat lunak, bukan manusia. Bot dapat melakukan apa saja, seperti mengajar, mencari, \textit{broadcast}, mengingatkan, menghubungkan, dan mengintegrasikan dengan layanan lainnya atau bahkan memberikan perintah ke \textit{Internet of Things}.\\
\tab Pengguna dapat berinteraksi dengan bot dengan mengirimkan pesan, perintah, dan permintaan. Sedangkan pengembang dapat mengontrol bot yang sedang dikembangkannya melalui \textit{HTTPS Requests} ke bot API Telegram. 


\cleardoublepage