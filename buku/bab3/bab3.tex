\chapter{TINJAUAN PUSTAKA}

\section{Laravel}
\tab Laravel adalah sebuah framework PHP yang dirilis dibawah lisensi MIT, dibangun dengan konsep MVC (model view controller). Laravel adalah pengembangan website berbasis MVP yang ditulis dalam PHP yang dirancang untuk meningkatkan kualitas perangkat lunak dengan mengurangi biaya pengembangan awal dan biaya pemeliharaan, dan untuk meningkatkan pengalaman bekerja dengan aplikasi dengan menyediakan sintaks yang ekspresif, jelas dan menghemat waktu\cite{laravel}.\\
\tab Dalam pengerjaan aplikasi Monitoring SIK ini, digunakan \textit{framework} laravel untuk memudahkan pembuatan website. Laravel merupakan sebuah \textit{framework} yang dapat mencukupi kebutuhan pengguna. Karena membutuhkan waktu yang singkat dalam pengerjaan (\textit{development}), serta mudah untuk \textit{maintenance} sistem.

\section{Javascript}
\tab Javascript adalah sebuah bahasa pemrograman tingkat tinggi yang dinamis yang terkenal dengan first class functionnya yang artinya javascript memperlakukan sebuah function dengan porsi yang sama dengan variabel lainnya. Contohnya adalah pada bahasa pemrograman javascript, sebuah function dapat dijadikan parameter dari function yang lainnya. Javascript merupakan bahasa scripting untuk halaman web yang paling populer saat ini. Selain lingkungan browser, banyak pula platform lain yang menggunakan bahasa javascript seperti node.js dan apache CouchDB\cite{javascript}.\\
\tab Dalam pengerjaan aplikasi Monitoring SIK, dibutuhkan beberapa halaman dengan fitur untuk menampilkan beberapa pilihan yang hanya bisa dipilih apabila memilih fitur tertentu, oleh karena itu digunakan bahasa javascript untuk menampilkan pilihan tersebut.

\section{PHP}
\tab PHP adalah bahasa pemrograman yang mengelola web service yang menggunakan protokol HTTP. Web Service ini dibuat agar bisa dipanggil atau diakses oleh aplikasi lain melalui internet dengan menggunakan format pertukaran data sebagai format pengiriman pesan. File PHP ini berisi query untuk mengolah database yang akan di proses pada aplikasi\cite{PHP}.

\section{Apache HTTP Server}
\tab Dalam pembuatan Sistem Monitoring SIK ini, digunakan Apache HTTP Server sebagai server yang menjalankan sistem yang telah dibuat. Apache HTTP Server merupakan sebuah \textit{paltform web server}. Apache mendukung beberapa fitur, beberapa diimplementasikan sebagai modul yang dikompilasi. Perangkat ini berkisar dari bahasa pemrograman sisi server yang mendukung skema otentikasi. Beberapa antarmuka bahasa yang mendukung antara lain Perl, Python, Tcl dan PHP\cite{apache}.

\section{MySQL }
\tab Database pada sistem ini menggunakan MySQL. MySQL adalah sebuah perangkat lunak sistem manajemen basis data SQL. Database MySQL mendukung beberapa fitur seperti \textit{multithread}, \textit{multi-user}, dan SQL database manajemen sistem(DMBMS). Database ini digunakan untuk keperluan sistem database yang cepat, handal dan mudah digunakan\cite{mysql}.