\chapter{PENDAHULUAN}

\section{Latar Belakang}
\tab PT Telekomunikasi Indonesia (Telkom) saat ini sedang mengembangkan produk barunya 



PT Bank Rakyat Indonesia (BRI) merupakan satu-satunya Bank di dunia yang memilik satelit sendiri. Satelit yang dinamakan BRIsat ini diluncurkan dan mengorbit sejak 9 Juni 2016 lalu. Adanya satelit ini bertujuan untuk menjangkau nasabah BRI di seluruh Indonesia. Khususnya di daerah-daerah terpencil.
\tab Sebelumnya, untuk penanganan kebutuhan komunikasi data dengan ribuan kantor wilayah, kantor cabang, KCP, BRI Unit, Teras BRI dan 22.792 ATMnya, BRI menyewa transponder satelit lain. Dengan adanya satelit milik BRI sendiri, mulai diadakan relokasi ATM yang akan diarahkan kepada satelitnya sendiri.\\
\tab Kegiatan relokasi ini dalam sehari mencapai 100 titik atau lebih. Saat ini, untuk pencatatan kegiatan relokasinya hanya dilakukan dengan menggunakan \textit{tools} berupa Microsoft Excel. Oleh karena itu, dibutuhkan suatu sistem untuk memonitoring kegiatan relokasi ATM menuju BRIsat\cite{bri}.

\section{Tujuan}
\tab Tujuan dari kerja praktik ini adalah untuk memenuhi kewajiban kuliah Kerja Praktik (KP) di Departemen Informatika Institut Teknologi Sepuluh Nopember dengan beban tiga SKS. Selain itu juga untuk memenuhi kebutuhan yang diperlukan oleh PT Telekomunikasi Indonesia dengan mengimplementasikan rancang bangun \sistem.\\ 
\tab Pembuatan \sistemSpasi ini bertujuan untuk:
\begin{enumerate}
	\item Membuat sebuah aplikasi berbasis \textit{Internet of Things} yang mudah dimengerti oleh pengguna dan dapat diakses dimana saja.
	\item Membuat sebuah aplikasi berbasis \textit{Internet of Things} yang dapat meningkatkan efisiensi persiapan ruang rapat.
\end{enumerate}

\section{Manfaat}
\tab Manfaat dari pembuatan \sistemSpasi ini adalah:
\begin{enumerate}
	\item Dapat membantu persiapan ruang rapat supaya lebih efektif dan efisien.
	\item Dapat mempersingkat waktu persiapan ruang rapat, sehingga rapat dapat dimulai tepat waktu.
\end{enumerate}

\section{Rumusan Masalah}
\tab Rumusan masalah dari kerja praktik ini adalah:
\begin{enumerate}
	\item Bagaimana membuat sebuah aplikasi berbasis \textit{Internet of Things} yang mudah dimengerti oleh pengguna dan dapat diakses dimana saja?
	\item Bagaimana membuat sebuah aplikasi berbasis \textit{Internet of Things} yang dapat meningkatkan efisiensi persiapan ruang rapat?
\end{enumerate}

\section{Lokasi dan Waktu Kerja Praktik}
\tab Lokasi kerja praktik berada di kantor \perusahaan dengan alamat \alamatPerusahaan - Daerah Khusus Ibukota Jakarta. Kami ditempatkan di Divisi \textit{Information Technology} (IT), Unit \textit{Service Platform Development} (SPD).\\
\tab Adapun kerja praktik dimulai pada tanggal 2 Januari 2018 hingga 2 Februari 2018, dengan hari kerja Senin sampai Jumat pukul 08.00 sampai dengan pukul 17.00 WIB dan sesuai dengan aturan-aturan perusahaan lainnya.

\section{Metodologi}
\tab Metodologi dalam pembuatan buku Kerja Praktik ini meliputi:
\begin{enumerate}
	\item \textbf{Perumusan Masalah}\\
	\tab Untuk mengetahui permasalahan apa yang harus diselesaikan, diberikn penjelasan mengenai alasan mengapa sistem ini dibutuhkan. Dijelaskan pula secara rinci mengenai alur sistem tersebut harus berjalan domain dan fungsionalitas, dijelaskan secara rinci bagaimana sistem yang harus dibuat. Penjelasan oleh pembimbing kerja praktik kali ini menghasilkan beberapa catatan mengenai gambaran cara kerja sistem dan rincian kebutuhan sistem. Setelah mendapatkan gambaran sistem, diskusi lebih lanjut dilakukan guna menentukan DBMS, bahasa pemrograman, dan framework yang dipakai dalam pembuatan sistem.\\
	\item \textbf{Studi Literatur}\\
	\tab Pada tahap ini, setelah ditentukannya DBMS, bahasa pemrograman sampai dengan framework yang digunakan, dilakukan studi literatur lanjut mengenai bagaimana penggunaannya dalam membangun sistem sesuai yang diharapkan.\\
	\tab Secara garis besar, untuk membuat Monitoring SIK digunakan bahasa pemrograman HTML, CSS, Javascript dan PHP untuk back end sistem, serta DBMS MySQL sebagai penyimpanan data relokasi, yang dikemas	melalui framework Laravel.\\
	\item \textbf{Analisis dan Perancangan Sistem}\\
	\tab Pada tahap ini akan dijelaskan tentang analisis serta perancangan sistem yang akan dibangun oleh penulis.\\
	\item \textbf{Implementasi Sistem}\\
	\tab Implementasi merupakan tahap pembangunan rancangan. Pada tahap ini merealisasikan apa yang terjadi pada tahap sebelumnya, sehingga sesuai dengan apa yang telah direncanakan.\\
	\item \textbf{Pengujian dan Evaluasi}\\
	\tab Pada tahap ini dilakukan uji coba pada aplikasi yang telah diimplementasikan. Tahap ini bermaksud untuk mengevaluasi kesesuaian sistem dan aplikasi yang dibuat apakah dapat dilakukan dengan lancar atau tidak. Selain itu juga untuk mencari masalah yang mungkin timbul dan tidak lupa mengadakan perbaikan jika terdapat kesalahan.\\
	\item \textbf{Kesimpulan dan Saran}\\
	\tab Pengujian yang dilakukan ini telah memenuhi syarat yang diinginkan, dan berjalan dengan baik dan lancar.
\end{enumerate}

\section{Sistematika Laporan}
Laporan Kerja Praktik ini terbagi menjadi 7 bab dengan rincian sebagai berikut:
	\begin{enumerate}
	\item BAB I: PENDAHULUAN
	
	Bab ini berisi latar belakang, tujuan, manfaat, rumusan masalah, lokasi dan waktu kerja praktik, metodologi dan sistematika laporan.
		
	\item BAB II: PROFIL PERUSAHAAN
		
	Bab ini berisi gambaran umum PT Bank Rakyat Indonesia, mulai dari sejarah, tujuan, visi dan misi perusahaan, dan divisi tempat kerja praktik dilakukan.
		
	\item BAB III: TINJAUAN PUSTAKA
		
	Bab ini berisi dasar teori dari metode/teknologi yang digunakan dalam menyelesaikan proyek kerja praktik.
		
	\item BAB IV: ANALISIS DAN PERANCANGAN SISTEM
		
	Bab ini dijelaskan mengenai desain antarmuka aplikasi.
		
	\item BAB V: IMPLEMENTASI SISTEM
		
	Bab ini berisi uraian tahap-tahap yang dilakukan untuk proses implementasi aplikasi.
		
	\item BAB VI: HASIL DAN UJI COBA
	
	Bab ini berisi hasil uji coba dan evaluasi dari perangkat lunak yang telah dikembangkan selama pelaksanaan kerja praktik.
	
	\item KESIMPULAN DAN SARAN
	
	Bab ini berisi kesimpulan dan saran yang didapat dari proses pelaksanaan kerja praktik.
	\end{enumerate}

\cleardoublepage
