\chapter{PENDAHULUAN}

\section{Latar Belakang}
\tab PT Telkom Indonesia (Persero) Tbk (Telkom) adalah Badan Usaha Milik Negara (BUMN) yang bergerak di bidang jasa layanan teknologi informasi dan komunikasi (TIK) dan jaringan telekomunikasi di Indonesia. Misi Telkom adalah "\textit{Lead Indonesian Digital Innovation and Globalization}",\cite{Profil Telkom} yang artinya Telkom harus memimpin inovasi digital di Indonesia, salah satunya adalah inovasi teknologi \textit{Internet of Things} yang belum banyak diimplementasikan di Indonesia. \\
\tab \textit{Internet of Things} (IoT) merupakan sebuah teknologi yang mampu mengubah perangkat menjadi sesuatu yang berharga, diantaranya untuk \textit{monitoring} dan analisis. Di Indonesia, teknologi IoT masih kalah dengan industri teknologi lainnya semacam \textit{e-commerce} dan teknologi finansial. Banyak hal yang menghambat pertumbuhan IoT di Indonesia, mulai dari kebijakan mengenai perangkat, data, hingga frekuensi penggunaan. Dari segi pemanfaatannya, IoT memiliki banyak peluang, baik untuk pengguna umum atau bisnis. Dari rilis yang dikeluarkan Hitachi, teknologi IoT akan menjadi tren global di tahun 2018.\cite{Perkembangan Industri Internet of Things di Indonesia Tahun 2017} \\
\tab Saat ini, Telkom sedang mengembangkan sebuah produk yang dapat mengimplementasikan \textit{server-side Internet of Things} (IoT) \textit{platform} bernama WSO2 IoT Server. Produk tersebut nantinya akan diperjualbelikan layaknya produk-produk Telkom yang lain, seperti IndiHome. Namun, Telkom harus menguji produk tersebut terlebih dahulu sebelum diluncurkan dan diperjualbelikan. \\
\tab Di sisi lain, Telkom merupakan sebuah perusahaan besar yang aktivitas para pegawainya tidak lepas dari kata "rapat". Rapat sudah menjadi bagian dari kehidupan sehari-hari karena ada banyak permasalahan yang harus didiskusikan dan juga sebagai media untuk melakukan \textit{sharing progress}. Namun, kami melihat persiapan rapat di Telkom kurang efisien. Ada banyak hal yang harus dilakukan, seperti menyalakan AC \textit{(Air Conditioner)}, menyalakan lampu, dan sebagainya, sehingga waktu dimulainya rapat menjadi molor. \\
\tab Dari beberapa permasalahan diatas, Telkom membutuhkan sebuah sistem yang dapat digunakan untuk mencoba produk baru Telkom, yaitu WSO2 IoT Server, sekaligus memiliki daya guna lebih, yakni untuk meningkatkan efisiensi persiapan ruang rapat di PT Telkom Indonesia. Oleh karena itu, kami menawarkan solusi dengan membangun sebuah sistem berbasis \textit{Internet of Things} yang bernama "EasyMeeting". Sistem ini memiliki konsep yang sama seperti \textit{Smart House}, yakni membuat sebuah ruangan pintar yang dapat dikendalikan menggunakan \textit{mobile phone}, \textit{device} yang dimiliki oleh hampir semua orang. \\
\tab Dengan adanya \namaSistem ini, diharapkan efisiensi rapat dan produktivitas di Telkom dapat meningkat. Sistem ini nantinya juga dapat digabungkan dengan sistem manajemen ruang rapat yang sudah ada di Telkom. \textit{EasyMeeting, meeting as easy as chatting}.  

\section{Tujuan}
\tab Tujuan dari pengerjaan kerja praktik ini adalah:
\begin{enumerate}
	\item Membangun sebuah sistem berbasis \textit{Internet of Things} yang dapat meningkatkan efisiensi persiapan ruang rapat di PT Telkom Indonesia sekaligus untuk menguji produk Telkom, WSO2 IoT Server.
	\item Membangun sebuah sistem berbasis \textit{Internet of Things} yang mudah dimengerti oleh pengguna dan dapat diakses dimana saja.
\end{enumerate}

\section{Manfaat}
\tab Manfaat yang diperoleh dari pengerjaan kerja praktik ini adalah:
\begin{enumerate}
	\item Meningkatkan efisiensi dan mempersingkat waktu persiapan ruang rapat di PT Telkom Indonesia, sehingga rapat dapat dimulai tepat waktu.
	\item Telkom dapat menguji produk barunya, yakni WSO2 IoT Server.
	\item Pegawai Telkom dapat mengontrol dan memonitor keadaan ruang rapat dengan mudah dan dimana saja.
	\item Menambah ilmu dalam pembuatan sistem berbasis \textit{Internet of Things} menggunakan teknologi-teknologi terbaru.
\end{enumerate}

\section{Rumusan Masalah}
\tab Rumusan masalah yang akan dibahas dalam pengerjaan kerja praktik ini adalah:
\begin{enumerate}
	\item Bagaimana membangun sebuah sistem berbasis \textit{Internet of Things} yang dapat meningkatkan efisiensi persiapan ruang rapat di PT Telkom Indonesia sekaligus untuk menguji produk Telkom, WSO2 IoT Server?
	\item Bagaimana membangun sebuah sistem berbasis \textit{Internet of Things} yang mudah dimengerti oleh pengguna dan dapat diakses dimana saja?
	
\end{enumerate}

\section{Lokasi dan Waktu Kerja Praktik}
\tab Lokasi pelaksanaan kerja praktik dilakukan di kantor \perusahaan dengan alamat \alamatPerusahaan - Daerah Khusus Ibukota Jakarta. Kami ditempatkan di Divisi \textit{Information Technology} (IT).\\
\tab Adapun kerja praktik dimulai pada tanggal 02 Januari 2018 hingga 02 Februari 2018, dengan hari kerja Senin sampai Jumat, pukul 08.00 sampai dengan pukul 17.00 WIB, dan sesuai dengan peraturan perusahaan.

\section{Metodologi}
\tab Tahapan metodologi dalam pengerjaan kerja praktik dapat dijabarkan sebagai berikut:
\begin{enumerate}
	\item \textbf{Perumusan Masalah}\\
	\tab Tahap pertama adalah merumuskan masalah. Untuk mengetahui permasalahan apa yang harus diselesaikan, diberikan penjelasan mengenai alasan mengapa sistem ini dibutuhkan. Dijelaskan pula secara rinci mengenai bagaimana alur sistem itu akan berjalan. Penjelasan yang diberikan oleh pembimbing kerja praktik menghasilkan beberapa catatan mengenai gambaran cara kerja sistem dan rincian kebutuhan sistem \textit{Internet of Things}. Setelah mendapatkan gambaran sistem, diskusi lebih lanjut dilakukan guna menentukan komponen-komponen \textit{microcontroller} yang dibutuhkan, bahasa pemrograman, dan \textit{platform} yang dipakai dalam pembuatan sistem.\\
	
	\item \textbf{Studi Literatur}\\
	\tab Setelah ditentukannya komponen-komponen \textit{microcontroller} yang dibutuhkan, bahasa pemrograman, dan \textit{platform} yang dipakai dalam pembuatan sistem, maka pada tahap ini dilakukan proses pencarian, pembelajaran, pengumpulan
	dan pemahaman informasi serta studi literatur lebih lanjut mengenai cara implementasinya dalam membangun sistem sesuai yang dibutuhkan.\\
	\tab Secara garis besar, untuk membuat \namaSistem digunakan beberapa komponen \textit{microcontroller}, yaitu nodeMCU, sensor suhu dan \textit{infrared}, serta \textit{jumper wires}, bahasa pemrograman C++, platform Telegram Bot API dan WSO2 IoT Server, serta DBMS H2 atau SQLite sebagai penyimpanan data pegawai Telkom.\\
	
	\item \textbf{Analisis dan Perancangan Sistem}\\
	\tab Pada tahap ini akan dijelaskan tentang analisis serta perancangan sistem yang akan dibangun oleh penulis, berdasarkan studi literatur dan pembelajaran konsep teknologi dari perangkat lunak yang sudah ada.\\
	\tab Fungsi dari sistem \textit{Internet of Things} yang akan dibuat adalah untuk meningkatkan efisiensi persiapan ruang rapat, meliputi menyalakan dan mematikan lampu dan AC (\textit{Air Conditioner}), menyesuaikan suhu ruangan, serta menampilkan status ruangan. Sistem IoT dikendalikan melalui aplikasi Telegram, dengan memanfaatkan Telegram Bot API. Sistem IoT hanya bisa dikendalikan oleh pegawai internal Telkom saja, sehingga sistem ini juga dilengkapi dengan \textit{user authentication} dan \textit{database} pegawai Telkom.\\
	
	\item \textbf{Implementasi Sistem}\\
	\tab Implementasi merupakan tahap pembangunan rancangan. Pada tahap ini, penulis merealisasikan sesuai dengan analisis dan perancangan sistem yang telah dilakukan pada tahap sebelumnya.\\
	\tab Setidaknya ada empat pekerjaan utama yang dilakukan, yaitu desain antarmuka pengguna, yakni memanfaatkan Telegram Bot API; desain antarmuka perangkat keras, yakni merangkai komponen-komponen \textit{microcontroller} yang dibutuhkan, meliputi NodeMCU, sensor suhu dan \textit{infrared}, serta \textit{jumper wires}; desain fungsi-fungsi yang bekerja dalam sistem (\textit{back-end}), yakni menggunakan bahasa C/C++ dan Arduino IDE; dan desain \textit{database}. Pembangunan sistem dilakukan selama satu bulan.\\ 
	
	\item \textbf{Pengujian dan Evaluasi}\\
	\tab Pada tahap ini dilakukan uji coba pada sistem yang telah dibangun. Tahap ini dimaksudkan untuk menguji dan mengevaluasi fitur-fitur dalam sistem yang telah dibuat, apakah berjalan dengan baik dan sesuai dengan kebutuhan atau tidak. Kesesuaian sistem dengan kebutuhan akan menentukan keberhasilan dalam pengujian. Selain itu, pengujian dan evaluasi juga dimaksudkan untuk mencari masalah yang mungkin timbul, serta mencari solusi untuk membuat sistem menjadi lebih baik.\\
	
	\item \textbf{Kesimpulan dan Saran}\\
	\tab Pada tahap ini dilakukan penarikan kesimpulan berdasarkan hasil implementasi dan pengujian sistem, serta diberikan saran terkait dengan hasil evaluasi untuk mengembangkan sistem menjadi lebih baik lagi.
\end{enumerate}

\section{Sistematika Laporan}
\tab Laporan kerja praktik ini terdiri dari tujuh bab dengan rincian sebagai berikut:
\begin{enumerate}
	\item \textbf{Bab I: Pendahuluan}\\
	\tab Bab ini memaparkan mengenai garis besar kerja praktik yang meliputi latar belakang, tujuan, manfaat, rumusan masalah, lokasi dan waktu kerja praktik, metodologi dan sistematika laporan.\\
		
	\item \textbf{Bab II: Profil Perusahaan}\\
	\tab Bab ini berisi gambaran umum profil perusahaan tempat pelaksanaan kerja praktik, yakni PT Telekomunikasi Indonesia (Persero) Tbk (Telkom), mulai dari visi dan misi perusahaan, portofolio bisnis perusahaan, sejarah perusahaan, hingga divisi tempat kerja praktik dilakukan, yakni divisi \textit{Information Technology} (IT).\\
		
	\item \textbf{Bab III: Tinjauan Pustaka}\\
	\tab Bab ini berisi penjelasan tentang istilah-istilah atau dasar teori dari metode atau teknologi yang digunakan dalam menyelesaikan proyek kerja praktik.\\
		
	\item \textbf{Bab IV: Analisis dan Perancangan Sistem}\\
	\tab Pada bab ini dijelaskan mengenai analisis terhadap sistem dan pemaparan mengenai kebutuhan untuk perancangan sistem yang akan dibangun dan dikembangkan.\\
	
	\item \textbf{Bab V: Implementasi Sistem}\\
	\tab Bab ini berisi uraian tahap-tahap implementasi sistem.\\
		
	\item \textbf{Bab VI: Pengujian dan Evaluasi}\\
	\tab Bab ini berisi hasil uji coba dan evaluasi dari sistem yang telah dikembangkan selama pelaksanaan kerja praktik.\\
	
	\item \textbf{Bab VII: Kesimpulan dan Saran}\\	
	\tab Bab ini berisi kesimpulan dan saran yang didapat dari proses pelaksanaan kerja praktik.\\
\end{enumerate}

\cleardoublepage
