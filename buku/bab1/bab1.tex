\chapter{PENDAHULUAN}

\section{Latar Belakang}
\tab PT Bank Rakyat Indonesia (BRI) merupakan satu-satunya Bank di dunia yang memilik satelit sendiri. Satelit yang dinamakan BRIsat ini diluncurkan dan mengorbit sejak 9 Juni 2016 lalu. Adanya satelit ini bertujuan untuk menjangkau nasabah BRI di seluruh Indonesia. Khususnya di daerah-daerah terpencil.
\tab Sebelumnya, untuk penanganan kebutuhan komunikasi data dengan ribuan kantor wilayah, kantor cabang, KCP, BRI Unit, Teras BRI dan 22.792 ATMnya, BRI menyewa transponder satelit lain. Dengan adanya satelit milik BRI sendiri, mulai diadakan relokasi ATM yang akan diarahkan kepada satelitnya sendiri.\\
\tab Kegiatan relokasi ini dalam sehari mencapai 100 titik atau lebih. Saat ini, untuk pencatatan kegiatan relokasinya hanya dilakukan dengan menggunakan \textit{tools} berupa Microsoft Excel. Oleh karena itu, dibutuhkan suatu sistem untuk memonitoring kegiatan relokasi ATM menuju BRIsat\cite{bri}.

\section{Tujuan}
Pembuatan Sistem Monitoring SIK ini bertujuan untuk:
\begin{enumerate}
\item Memudahkan monitoring kegiatan relokasi ATM
\item Melakukan efisiensi kerja 
\end{enumerate}

\section{Manfaat}
Manfaat dari pembuatan Sistem Monitoring SIK ini adalah sebagai berikut:

\begin{enumerate}
	\item Membuat pengerjaan monitoring relokasi ATM lebih efisien
	\item Membantu memudahkan kegiatan monitoring relokasi ATM
\end{enumerate}

\section{Rumusan Masalah}
Rumusan Masalah dari Kerja Praktik ini adalah sebagai berikut:

\begin{enumerate}
	\item Bagaimana menciptakan aplikasi web monitoring yang mudah dimengerti oleh pengguna?
	\item Bagaimana menciptakan aplikasi web monitoring yang dapat membantu efisiensi pengerjaan relokasi ATM?
\end{enumerate}

\section{Lokasi dan Waktu Kerja Praktik}
\tab Lokasi kerja praktik berada di PT. Bank Rakyat Indonesia dengan alamat Jalan RM. Harsono, RT 6/ RW 7, Ragunan, Pasar Minggu, Jakarta Selatan.\\
\tab Adapun kerja praktik dimulai pada tanggal 14 Juni 2017 hingga 9 Agustus 2017 dengan hari kerja Senin sampai Jumat pukul 07.30 sampai dengan pukul 16.30 WIB (8 jam kerja dan 1 jam istirahat).

\section{Metodologi}
Metodologi dalam pembuatan buku Kerja Praktik ini meliputi:
\begin{enumerate}
	\item \textbf{Perumusan Masalah}\\
	Untuk mengetahui domain dan fungsionalitas, dijelaskan secara rinci bagaimana sistem yang harus dibuat. Penjelasan oleh pembimbing kerja praktik kali ini menghasilkan beberapa catatan mengenai gambaran cara kerja sistem dan rincian kebutuhan sistem. Setelah mendapatkan gambaran sistem, diskusi lebih lanjut dilakukan guna menentukan DBMS, bahasa pemrograman, dan framework yang dipakai dalam pembuatan sistem.
	
	\item \textbf{Studi Literatur}\\
	Pada tahap ini, setelah ditentukannya DBMS, bahasa pemrograman sampai dengan framework yang digunakan, dilakukan studi literatur lanjut mengenai bagaimana penggunaannya dalam membangun sistem sesuai yang diharapkan.
	
	Secara garis besar, untuk membuat Monitoring SIK digunakan bahasa pemrograman HTML, CSS, Javascript dan PHP untuk back end sistem, serta DBMS MySQL sebagai penyimpanan data relokasi, yang dikemas	melalui framework Laravel.
	
	\item \textbf{Analisis dan Perancangan Sistem}\\
	Pada tahap ini akan dijelaskan tentang analisis serta perancangan sistem yang akan dibangun oleh penulis.
	
	\item \textbf{Implementasi Sistem}\\
	Implementasi merupakan tahap pembangunan rancangan. Pada tahap ini merealisasikan apa yang terjadi pada tahap sebelumnya, sehingga sesuai dengan apa yang telah direncanakan.
	
	\item \textbf{Pengujian dan Evaluasi}\\
	Pada tahap ini dilakukan uji coba pada aplikasi yang telah diimplementasikan. Tahap ini bermaksud untuk mengevaluasi kesesuaian sistem dan aplikasi yang dibuat apakah dapat dilakukan dengan lancar atau tidak. Selain itu juga untuk mencari masalah yang mungkin timbul dan tidak lupa mengadakan perbaikan jika terdapat kesalahan.
	
	\item \textbf{Kesimpulan dan Saran}\\
	Pengujian yang dilakukan ini telah memenuhi syarat yang diinginkan, dan berjalan dengan baik dan lancar.
\end{enumerate}

\section{Sistematika Laporan}
Laporan Kerja Praktik ini terbagi menjadi 7 bab dengan rincian sebagai berikut:
	\begin{enumerate}
	\item BAB I: PENDAHULUAN
	
	Bab ini berisi latar belakang, tujuan, manfaat, rumusan masalah, lokasi dan waktu kerja praktik, metodologi dan sistematika laporan.
		
	\item BAB II: PROFIL PERUSAHAAN
		
	Bab ini berisi gambaran umum PT Bank Rakyat Indonesia, mulai dari sejarah, tujuan, visi dan misi perusahaan, dan divisi tempat kerja praktik dilakukan.
		
	\item BAB III: TINJAUAN PUSTAKA
		
	Bab ini berisi dasar teori dari metode/teknologi yang digunakan dalam menyelesaikan proyek kerja praktik.
		
	\item BAB IV: ANALISIS DAN PERANCANGAN SISTEM
		
	Bab ini dijelaskan mengenai desain antarmuka aplikasi.
		
	\item BAB V: IMPLEMENTASI SISTEM
		
	Bab ini berisi uraian tahap-tahap yang dilakukan untuk proses implementasi aplikasi.
		
	\item BAB VI: HASIL DAN UJI COBA
	
	Bab ini berisi hasil uji coba dan evaluasi dari perangkat lunak yang telah dikembangkan selama pelaksanaan kerja praktik.
	
	\item KESIMPULAN DAN SARAN
	
	Bab ini berisi kesimpulan dan saran yang didapat dari proses pelaksanaan kerja praktik.
	\end{enumerate}

\cleardoublepage
