\chapter{KESIMPULAN DAN SARAN}

\section{Kesimpulan}
\tab Kesimpulan yang didapat setelah melakukan pengembangan sistem EasyMeeting pada kegiatan Kerja Praktik di PT Telekomunikasi Indonesia adalah sebagai berikut:
\begin{enumerate}
\item Sistem EasyMeeting berhasil dibuat menggunakan NodeMCU, Telegram Bot API, dan bahasa pemrograman Python untuk meningkatkan efisiensi persiapan ruang rapat di PT Telkom Indonesia.
\item Sistem EasyMeeting dapat membantu meningkatkan pengembangan \textit{Internet of Things} (IoT) di PT Telkom Indonesia.
\item Sistem EasyMeeting berhasil dibuat sesuai dengan analisis dan perancangan sistem, serta memenuhi kebutuhan
\end{enumerate}

\section{Saran}
Setelah melalui Kerja Praktik kali ini, penulis memberikan saran sebagai berikut:
\begin{enumerate}
\item Tampilan \textit{front end} harus teroptimasi dengan baik pada seluruh \textit{device} dan \textit{browser}. Mengingat beragamnya \textit{device} dan \textit{browser} yang digunakan pengguna.
\item Perlu dilakukan penelitian terhadap segala pengembangan fitur sistem di masa mendatang.
\end{enumerate}