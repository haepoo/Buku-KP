\chapter{KESIMPULAN DAN SARAN}

\section{Kesimpulan}
\tab Kesimpulan yang didapat setelah melakukan pengembangan sistem EasyMeeting pada kegiatan Kerja Praktik di PT Telekomunikasi Indonesia adalah sebagai berikut:
\begin{enumerate}
\item \sistem berhasil dibuat menggunakan Arduino UNO, Telegram Bot API, dan bahasa pemrograman C untuk meningkatkan efisiensi persiapan ruang rapat di PT Telkom Indonesia.
\item \sistem dapat membantu meningkatkan pengembangan \textit{Internet of Things} (IoT) di PT Telkom Indonesia.
\item \sistem berhasil dibuat sesuai dengan analisis dan perancangan sistem, serta memenuhi kebutuhan fungsional dan non-fungsional PT Telkom Indonesia.
\end{enumerate}

\section{Saran}
Setelah melalui Kerja Praktik kali ini, penulis memberikan saran sebagai berikut:
\begin{enumerate}
\item Mikrokontroler Arduino UNO dan ESP8266-01 dapat diganti dengan mikrokontroler NodeMCU (board Arduino sekaligus ESP8266 yang telah ter-\textit{package} menjadi satu) supaya koneksi Wi-Fi pada sirkuit menjadi lebih stabil.
\item Perlu dilakukan penelitian lagi terhadap WSO2 IoT Server.
\end{enumerate}