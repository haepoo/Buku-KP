\var{\judul}{Rancang Bangun Sistem Berbasis Internet of Things "EasyMeeting" Menggunakan NodeMCU dan Telegram Bot Untuk Meningkatkan Efisiensi Persiapan Ruang Rapat }
\var{\perusahaan}{PT Telekomunikasi Indonesia Tbk (Telkom) STO Gambir }
\var{\alamatPerusahaan}{Jalan Medan Merdeka Selatan No. 12 Jakarta Pusat }
\var{\kodePos}{10110 }
\var{\periode}{02 Januari-02 Februari 2018 }

\var{\namaPenulisSatu}{Hafara Firdausi }
\var{\nrpPenulisSatu}{5115 100 043 }
\var{\namaPenulisDua}{Nahda Fauziyah Zahra }
\var{\nrpPenulisDua}{5115 100 141 }
\var{\departemen}{Informatika }
\var{\fakultas}{Teknologi Informasi dan Komunikasi }
\var{\prodi}{S-1 }
\var{\perguruanTinggi}{Institut Teknologi Sepuluh Nopember }
\var{\pembimbingDept}{Radityo Anggoro S.Kom, M.Sc }
\var{\nipPembimbingDept}{19841016 200812 1 002 }
\var{\pembimbingLap}{Doddy Nur Pratomo }
\var{\nipPembimbingLap}{790101 }
\var{\jabatanPembimbingLap}{Unit \textit{Service Platform Development} }

\var{\sistem}{Sistem Berbasis \textit{Internet of Things} "EasyMeeting"}
\var{\namaSistem}{sistem berbasis \textit{Internet of Things} "EasyMeeting" }