\chapter{PROFIL PERUSAHAAN}
\section{Tentang Perusahaan}
\tab PT Telkom Indonesia (Persero) Tbk (Telkom) adalah Badan Usaha Milik Negara (BUMN) yang bergerak di bidang jasa layanan teknologi informasi dan komunikasi (TIK) dan jaringan telekomunikasi di Indonesia. Pemegang saham mayoritas Telkom adalah Pemerintah Republik Indonesia sebesar 52.09\%, sedangkan 47.91\% sisanya dikuasai oleh publik. Saham Telkom diperdagangkan di Bursa Efek Indonesia (BEI) dengan kode “TLKM” dan New York Stock Exchange (NYSE) dengan kode “TLK” \cite{profil-telkom}.

\section{Visi dan Misi Perusahaan}
\tab Seiring dengan perkembangan teknologi digital dan transformasi perusahaan, Telkom memiliki visi dan misi baru yang diberlakukan sejak 2016, yaitu:
\subsection{Visi Perusahaan}
\tab Visi Telkom adalah "\textit{Be the King of Digital in the Region}", yang artinya Telkom akan menjadi perusahaan yang unggul dalam bidang teknologi digital \textit{(king of digital)} di kawasan regional.
\subsection{Misi Perusahaan}
\tab Misi Telkom adalah "\textit{Lead Indonesian Digital Innovation and Globalization}", yang artinya Telkom akan memimpin inovasi digital dan globalisasi di Indonesia. Untuk mencapai misi tersebut, Telkom memiliki \textit{strategic objectives} "\textit{Top 10 Market Capitalization Telco in Asia-Pacific by 2020 and maintain its stronghold position}".\\
\tab TelkomGroup juga telah menyusun strategi korporasi guna menciptakan \textit{sustainable competitive growth} dan mendorong cita-cita Indonesia untuk menjadi kekuatan ekonomi digital terbesar di Asia Tenggara, yaitu:
\begin{enumerate}
	\item \textit{Directional Strategy: Disruptive competitive growth}\\
	\tab Di tengah perubahan lingkungan industri yang sangat menantang, TelkomGroup yakin bahwa kapitalisasi pasar akan tumbuh secara signifikan. Ini dilakukan dengan cara memberikan nilai lebih kepada pelanggan melalui inovasi produk dan layanan, mendorong sinergi serta membangun ekosistem digital yang kuat baik di pasar domestik maupun internasional.\\
	\item \textit{Portfolio Strategy: Customer value through digital TIMES portfolio}\\
	\tab TelkomGroup berfokus pada portofolio digital TIMES \textit{(Telecommunication, Information, Media, Edutainment \& Services)} melalui penyediaan layanan yang nyaman dan konvergen sehingga memberikan nilai yang tinggi kepada pelanggan.\\
	\item \textit{Parenting Strategy: Strategic Control}\\
	\tab Untuk mendukung pertumbuhan bisnis secara efektif, TelkomGroup menerapkan pendekatan \textit{strategic control} untuk menyelaraskan unit bisnis, unit fungsional dan anak perusahaan agar proses dapat berjalan lebih terarah, bersinergi, dan efektif dalam mencapai tujuan perusahaan.
\end{enumerate}
	
\section{Portofolio Bisnis Perusahaan}
\tab Dalam upaya bertransformasi menjadi \textit{digital telecommunication company}, TelkomGroup mengimplementasikan strategi bisnis dan operasional perusahaan yang berorientasi kepada pelanggan \textit{(customer-oriented)}. Kegiatan usaha TelkomGroup bertumbuh dan berubah seiring dengan perkembangan teknologi, informasi dan digitalisasi, namun masih dalam koridor industri telekomunikasi dan informasi.\\
\tab Saat ini, TelkomGroup mengelola 6 produk portofolio yang melayani empat segmen konsumen, yaitu korporat, perumahan, perorangan dan segmen konsumen lainnya.\\
\tab Berikut penjelasan portofolio bisnis TelkomGroup:
\begin{enumerate}
	\item \textit{Mobile}\\
	\tab Portofolio ini menawarkan produk \textit{mobile voice}, SMS dan \textit{value added service}, serta \textit{mobile broadband}. Produk tersebut ditawarkan melalui entitas anak, Telkomsel, dengan merk  Kartu Halo untuk pasca bayar dan simPATI, Kartu As dan Loop untuk pra bayar.\\
	\item \textit{Fixed}\\
	\tab Portofolio ini memberikan layanan \textit{fixed service}, meliputi \textit{fixed voice, fixed broadband, }termasuk Wi-Fi dan \textit{emerging wireless technology} lainnya, dengan \textit{brand} IndiHome.\\
	\item \textit{Wholesale \& International}\\
	\tab Produk yang ditawarkan antara lain layanan interkoneksi, \textit{network service}, Wi-Fi, VAS, \textit{hubbing data center} dan \textit{content platform}, data dan internet, dan \textit{solution}.\\
	\item \textit{Network Infrastructure}\\
	\tab Produk yang ditawarkan meliputi \textit{network service}, satelit, infrastruktur dan \textit{tower}.\\
	\item \textit{Enterprise Digital}\\
	\tab Terdiri dari layanan \textit{information and communication technology platform service} dan \textit{smart enabler platform service}.\\
	\item \textit{Consumer Digital}\\
	\tab Terdiri dari media dan \textit{edutainment service}, seperti \textit{e-commerce} (blanja.com), video/TV dan \textit{mobile based digital service}. Selain itu, TelkomGroup juga menawarkan \textit{digital life service} seperti \textit{digital life style} (Langit Musik dan VideoMax), \textit{digital payment} seperti TCASH, \textit{digital advertising and analytics} seperti bisnis \textit{digital advertising} dan solusi \textit{mobile banking} serta \textit{enterprise digital service} yang menawarkan layanan \textit{Internet of Things (IoT)}.
\end{enumerate}

\section{Sejarah Perusahaan}
\tab Dalam perjalanan sejarahnya, Telkom telah melalui berbagai dinamika bisnis dan melewati beberapa fase perubahan, yakni:
\begin{enumerate}
	\item Kemunculan Telepon (1882)\\
	\tab Pada tahun 1856, pemerintah kolonial Belanda membangun telegraf elektromagnetik pertama yang menghubungkan Jakarta (Batavia) dan Bogor. Kemudian tahun 1882, kemunculan teknologi telepon menyaingi layanan pos dan telegraf. Pemerintah Belanda membentuk lembaga yang mengatur layanan pos dan telekomunikasi di Indonesia dengan nama Jawatan \textit{Post Telegraaf Telefoon} (PTT).\\
	\item Kelahiran Telkom (1965)\\
	\tab Pada tahun 1957, banyak perusahaan-perusahaan Belanda yang diakuisisi oleh Indonesia. Pemerintah Soekarno memiliki visi menjadikan seluruh perusahaan negara menjadi \textit{public corporation}. Pada tahun 1961, Jawatan PTT berubah nama menjadi Perusahaan Negara Pos dan Telekomunikasi (PN Postel).\\
	\tab Namun, seiring berkembangnya layanan telepon dan telex, pemerintah Indonesia mengeluarkan Peraturan Pemerintah No. 30 tanggal 6 Juli 1965 yang berisi pemisahan industri pos dan telekomunikasi dalam PN Postel menjadi PN Pos dan Giro serta PN Telekomunikasi.\\
	\tab Dengan pemisahan ini, setiap perusahaan dapat fokus untuk mengelola portofolio bisnisnya masing-masing. Terbentuknya PN Telekomunikasi ini menjadi cikal-bakal Telkom saat ini. Sejak tahun 2016, manajemen Telkom menetapkan tanggal 6 Juli 1965 sebagai hari lahir Telkom.\\
	\item Perumtel (1974)\\
	\tab Pada tahun 1974, PN Telekomunikasi berubah nama menjadi Perusahaan Umum Telekomunikasi Indonesia (Perumtel) yang menyelenggarakan jasa telekomunikasi nasional maupun internasional.\\
	\item PT Telekomunikasi Indonesia (1991)\\
	\tab Pada tahun 1991, Perumtel berubah bentuk menjadi Perseroan Terbatas (PT) dengan nama Telekomunikasi Indonesia (Telkom) berdasarkan Peraturan Pemerintah Nomor 25 Tahun 1991.\\
	\item Melakukan Penawaran Umum Perdana Saham Telkom (1995)\\
	\tab Pada tanggal 14 November 1995, dilakukan Penawaran Umum Perdana saham Telkom. Sejak itu saham Telkom tercatat dan diperdagangkan di Bursa Efek Jakarta (BEJ/JSX) dan Bursa Efek Surabaya (BES/SSX) (keduanya sekarang bernama Bursa Efek Indonesia (BEI/IDX), Bursa Efek New York (NYSE) (Diperdagangkan pada tanggal 14 Juli 2003) dan Bursa Efek London (LSE). Saham Telkom juga diperdagangkan tanpa pencatatan di Bursa Saham Tokyo. Jumlah saham yang dilepas saat itu adalah 933 juta lembar saham.\\
	\tab Sejak 16 Mei 2014, saham Telkom tidak lagi diperdagangkan di Bursa Efek Tokyo (TSE) dan pada 5 Juni 2014 di Bursa Efek London (LSE).\\
	\item Membeli Telkomsel (2001)\\
	\tab Tahun 2001, Telkom membeli 35\% saham Telkomsel dari PT Indosat sebagai bagian dari implementasi restrukturisasi industri jasa telekomunikasi di Indonesia yang ditandai dengan penghapusan kepemilikan bersama dan kepemilikan silang antara Telkom dan Indosat (duopoli).\\
	\item \textit{New Telkom} (2009)\\
	\tab Pada 23 Oktober 2009, Telkom meluncurkan \textit{"New Telkom"} ("Telkom baru") yang ditandai dengan penggantian identitas perusahaan.\\
	\item Beroperasi di Tujuh Negara (2013)\\
	\tab Telkom telah beroperasi di tujuh negara seperti Australia, Hong Kong, Macau, Timor Leste, Malaysia, Taiwan, dan Amerika Serikat.\\
	\item Meluncurkan Layanan 4G (2014)\\
	\tab Melalui anak perusahaan, Telkomsel, menjadi operator pertama yang meluncurkan layanan 4G secara komersial.
\end{enumerate}

\section{Divisi \textit{Information Technology}}
\tab Telkom memiliki struktur organisasi yang sangat besar dan kompleks, terdiri dari banyak divisi dan sub-divisi, salah satunya adalah divisi IT atau \textit{Information Technology}. Secara garis besar, divisi IT dibagi menjadi 2 bagian, yaitu \textit{Development} dan \textit{Operation}, dengan masing-masing bagian dipimpin oleh seorang \textit{Deputy EGM IT}. \textit{Deputy EGM IT Development} membawahi beberapa sub-divisi berikut:
\begin{enumerate}
	\item \textit{Enterprise \& Analytic Platform Development}
	\item \textit{BSS \& CEM Platform Development}
	\item \textit{OSS Platform Development}
	\item \textit{Service Platform Development}
	\item \textit{IT Planning \& Architecture}
\end{enumerate}
\tab Sedangkan \textit{Deputy EGM IT Operation} membawahi sub-divisi berikut:
\begin{enumerate}
	\item \textit{Enterprise \& Analytic Platform Operation}
	\item \textit{BSS \& CEM Platform Operation}
	\item \textit{OSS Platform Operation}
	\item \textit{Service Platform Operation}
	\item \textit{CFU/FU Support \& ITSM}
\end{enumerate}
\tab Kedua \textit{deputy} bekerja dibawah pimpinan seorang \textit{Executive General Manager Information Technology} yang langsung membawahi sub-divisi \textit{General Affair}.\\
\tab Pada kesempatan kerja praktik ini, kami ditempatkan di sub-divisi \textit{Service Platform Development} (SPD) pada unit \textit{SOA Platform Development}.

\cleardoublepage