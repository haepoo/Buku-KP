\chapter{PROFIL PERUSAHAAN}
\tab Sejak 1 Agustus 1992 berdasarkan Undang-Undang Perbankan No. 7 tahun 1992 dan Peraturan Pemerintah RI No. 21 tahun 1992 status BRI berubah menjadi perseroan terbatas. Kepemilikan BRI saat itu masih 100\% di tangan Pemerintah Republik Indonesia. Pada tahun 2003, Pemerintah Indonesia memutuskan untuk menjual 30\% saham bank ini, sehingga menjadi perusahaan publik dengan nama resmi PT. Bank Rakyat Indonesia (Persero) Tbk., yang masih digunakan sampai dengan saat ini\cite{bri}.
\section{Visi, Misi dan Tujuan Perusahaan}
Visi, Misi dan Tujuan dari BRI adalah sebagai berikut:
	\subsection{Visi Perusahaan}
	Visi Bank BRI yakni menjadi sebuah bank terkemuka di Indonesia yang akan selalu mengutamakan kepuasan para nasabahnya.
	\subsection{Misi Perusahaan}
	Misi Bank BRI adalah:
	\begin{enumerate}
		\item Bank BRI mampu melakukan segala jenis kegiatan perbankan terbaik dengan mengutamakan pelayanan yang diberikan kepada badan usaha mikro, menengah, dan kecil guna meningkatkan perekonomian masyarakat.
		\item Bank BRI akan senantiasa memberikan pelayanan prima pada setiap nasabahnya melalui jaringan BRI yang luas dan didukung dengan adanya sumber daya manusia professional serta teknologi yang handal, melaksanakan manajemen resiko dan praktek GCG (\textit{Good Coorporate Governance}) yang baik.
		\item Bank BRI akan memberikan keuntungan serta manfaat secara optimal pada pihak-pihak yang berkepentingan
	\end{enumerate}
	\subsection{Tujuan Perusahaan}
	Tujuan Bank BRI adalah:
	\begin{enumerate}
		\item Menjadi bank sehat dan salah satu dari lima bank terbesar dalam aset dan keuntungan.
		\item Menjadi bank terbesar dan terbaik dalam pengembangan usaha mikro, kecil dan menengah.
		\item Menjadi bank terbesar dan terbaik dalam pengembangan agrobisnis.
		\item Menjadi salah satu bank \textit{go public} terbaik.
		\item Menjadi bank yang melaksanakan \textit{Good Coorporate Governance} secara konsisten.
		\item Menjadikan budaya kerja BRI sebagai sikap dan perilaku semua insane BRI.
	\end{enumerate}

\section{Sejarah Perusahaan}
\tab Bank Rakyat Indonesia (BRI) adalah salah satu bank milik pemerintah yang terbesar di Indonesia. Pada awalnya Bank Rakyat Indonesia (BRI) didirikan di Purwokerto, Jawa Tengah oleh Raden Bei Aria Wirjaatmadja dengan nama De Poerwokertosche Hulp en Spaarbank der Inlandsche Hoofden atau "Bank Bantuan dan Simpanan Milik Kaum Priyayi Purwokerto", suatu lembaga keuangan yang melayani orang-orang berkebangsaan Indonesia (pribumi). Lembaga tersebut berdiri tanggal 16 Desember 1895, yang kemudian dijadikan sebagai hari kelahiran BRI.\\
\tab Pada periode setelah kemerdekaan RI, berdasarkan Peraturan Pemerintah No. 1 tahun 1946 Pasal 1 disebutkan bahwa BRI adalah sebagai Bank Pemerintah pertama di Republik Indonesia. Dalam masa perang mempertahankan kemerdekaan pada tahun 1948, kegiatan BRI sempat terhenti untuk sementara waktu dan baru mulai aktif kembali setelah perjanjian Renville pada tahun 1949 dengan berubah nama menjadi Bank Rakyat Indonesia Serikat. Pada waktu itu melalui PERPU No. 41 tahun 1960 dibentuklah Bank Koperasi Tani dan Nelayan (BKTN) yang merupakan peleburan dari BRI, Bank Tani Nelayan dan Nederlandsche Maatschappij (NHM). Kemudian berdasarkan Penetapan Presiden (Penpres) No. 9 tahun 1965, BKTN diintegrasikan ke dalam Bank Indonesia dengan nama Bank Indonesia Urusan Koperasi Tani dan Nelayan.\\
\tab Setelah berjalan selama satu bulan, keluar Penpres No. 17 tahun 1965 tentang pembentukan bank tunggal dengan nama Bank Negara Indonesia. Dalam ketentuan baru itu, Bank Indonesia Urusan Koperasi, Tani dan Nelayan (eks BKTN) diintegrasikan dengan nama Bank Negara Indonesia unit II bidang Rural, sedangkan NHM menjadi Bank Negara Indonesia unit II bidang Ekspor Impor (Exim).\\
\tab Berdasarkan Undang-Undang No. 14 tahun 1967 tentang Undang-undang Pokok Perbankan dan Undang-undang No. 13 tahun 1968 tentang Undang-undang Bank Sentral, yang intinya mengembalikan fungsi Bank Indonesia sebagai Bank Sentral dan Bank Negara Indonesia Unit II Bidang Rular dan Ekspor Impor dipisahkan masing-masing menjadi dua Bank yaitu Bank Rakyat Indonesia dan Bank Ekspor Impor Indonesia. Selanjutnya berdasarkan Undang-undang No. 21 tahun 1968 menetapkan kembali tugas-tugas pokok BRI sebagai bank umum.\\
\tab Sejak 1 Agustus 1992 berdasarkan Undang-Undang Perbankan No. 7 tahun 1992 dan Peraturan Pemerintah RI No. 21 tahun 1992 status BRI berubah menjadi perseroan terbatas. Kepemilikan BRI saat itu masih 100\% di tangan Pemerintah Republik Indonesia. Pada tahun 2003, Pemerintah Indonesia memutuskan untuk menjual 30\% saham bank ini, sehingga menjadi perusahaan publik dengan nama resmi PT. Bank Rakyat Indonesia (Persero) Tbk., yang masih digunakan sampai dengan saat ini\cite{bri}.

\section{Divisi Satelit}
\tab Pada kesempatan kali ini, penulis ditempatkan pada divisi satelit. Di divisi ini, berhubungan dengan penanganan satelit dan BRIsat yang digunakan pada mesin ATM BRI. Di sini, penulis berkesempatan membuat Sistem Informasi untuk Memonitoring Kegiatan Relokasi BRIsat BRI.

\cleardoublepage